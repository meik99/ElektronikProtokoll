%! Author = michael
%! Date = 08.10.20

% Preamble
\documentclass[11pt]{article}
\usepackage{pdfpages}
\usepackage[german]{babel}
\usepackage{amsmath}
\usepackage[backend=biber]{biblatex}
\usepackage{csquotes}
\usepackage{changepage}
\usepackage{graphicx}
\usepackage{listings}
\usepackage{xcolor}

\definecolor{codegreen}{rgb}{0,0.6,0}
\definecolor{codegray}{rgb}{0.5,0.5,0.5}
\definecolor{codepurple}{rgb}{0.58,0,0.82}
\definecolor{backcolour}{rgb}{0.95,0.95,0.92}

\lstdefinestyle{code}{
backgroundcolor=\color{backcolour},
commentstyle=\color{codegreen},
keywordstyle=\color{magenta},
numberstyle=\tiny\color{codegray},
stringstyle=\color{codepurple},
basicstyle=\ttfamily\footnotesize,
breakatwhitespace=false,
breaklines=true,
captionpos=b,
keepspaces=true,
numbers=left,
numbersep=5pt,
showspaces=false,
showstringspaces=false,
showtabs=false,
tabsize=2
}

\lstset{style=code}


\setlength{\parindent}{0em}
\setlength{\parskip}{0.8em}

\addbibresource{main.bib}

% Document
\begin{document}
    \includepdf[pages=-]{deckblatt.pdf}

    \tableofcontents
    \newpage


    \section{Einleitung}
    \label{sec:einleitung}

    Das Wissen um die Funktionsweise eines Mikrocontrollers ist wesentlicher Bestandteil eines Informatik-Studiums.
    Im Zuge Dieses werden in diesem Protokoll verschiedene Experimente beschrieben und evaluiert.
    Insgesamt werden sieben Experimente begutachtet und diskutiert, welche im Zeitraum vom 28.September 2020 und 2.Oktober 2020 durchgeführt wurden.


    \section{Verwendete Materialien}
    \label{sec:verwendete-materialien}

    Jede Sektion gibt die in ihr verwendeten Materialien mitsamt ihrer Stückzahl an.
    In dieser Sektion werden alle Materialien im Allgemeinen gelistet und beschrieben.

    \subsection{LED}
    \label{subsec:led}
    Eine Leuchtdiode (LED) ist ein Bauelement der Elektronik.
    Wird sie von elektrischen Strom durchflossen beginnt sie Licht auszustrahlen.
    Die Wellenlänge, d.h., die Farbe des Lichts sowie ob es für das menschliche Auge sichtbar ist oder nicht, hängt von den benutzten Materialien im inneren der LED ab.
    Die in der LED verwendeten Materialien sind für dieses Protokoll nicht weiter von Bedeutung.
    Für die beschriebenen Aufgaben wurden die folgenden LEDs verwendet.

    \begin{table}[h]
        \centering
        \caption{LEDs - Farben, Flussspannung, Maximalstrom \cite{led-elektrische-eigenschaften}}
        \label{tab:leds-farben-und-elemente}
        \begin{tabular}{| l | l | l |}
            \hline
            Farbe & Durchflussspannung & Maximalstrom \\
            \hline
            Rot   & 1,6V - 2,2V        & 20mA         \\
            Gelb  & 1,9V - 2,5V        & 20mA         \\
            Grün  & 1,9V - 2,5V        & 20mA         \\
            \hline
        \end{tabular}
    \end{table}

    \subsection{Widerstände}
    \label{subsec:widerstände}

    Widerstände werden verwendet, um einen Spannungsabfall in einem Stromkreis zu verursachen.
    Damit kann die Stromstärke in einem Stromkreis begrenzt beziehungsweise verringert werden.
    Oft wird bei der Verwendung von LEDs ein sogenannter Vorwiederstand verwendet, um die Stromstärke soweit abzusenken, dass die LED nicht beschädigt wird.

    In den Aufgaben wurden Widerstände der sogenannten E6-Reihe verwendet.
    E-Reihen sind normierte Widerstandsgrößen, wobei die Zahl die Stufen zwischen den Potenzen angibt.
    D.h., bei der E6-Reihe sind sechs verschiedene Widerstandsgrößen zwischen $10\Omega$ und $100\Omega$, zwischen $100\Omega$ und $1000\Omega$, u.s.w. bis zum oberen Limit von $10M\Omega$.

    Die verwendeten Widerstände sind Farbcodiert, um sie voneinander unterscheiden zu können.
    Die Farbkodierung von Widerständen wird in den Aufgaben, in denen sie verwendet werden, angegeben.
    Die Bedeutung der Codierung der verwendeten Widerstände kann in Tab. \ref{tab:farbcodierung-von-widerständen} abgelesen werden.

    \begin{table}
        \begin{adjustwidth}{-2cm}{}
            \caption{Farbcodierung von Widerständen}
            \label{tab:farbcodierung-von-widerständen}
            \begin{tabular}{| l | l | l | l | l |}
                \hline
                Farbe & 1.Ring (10er Stelle) & 2.Ring (1er Stelle) & 3.Ring (Multiplikator) & 4.Ring (Toleranz) \\
                \hline
                Braun & 1                    & 0                   & 10                     & $\pm 1$           \\
                Grün  & 5                    & 5                   & $100\ 000$             & $\pm 0.5$         \\
                Gold  & -                    & -                   & $0.1$                  & $\pm 5$           \\
                \hline
            \end{tabular}
        \end{adjustwidth}
    \end{table}

    \subsection{Mikrocontroller \cite{arduino-r3}}
    \label{subsec:mikrocontroller}

    Mikrocontroller werden verwendet, um komplexere Logik in eine elektronische Schaltung zu integrieren.
    Die in diesem Protokoll verwendete Mikrocontroller sind Mikrocontroller vom Typ Arduino Uno Rev 3.
    Diese Mikrocontroller besitzen 14 digitale Pins, an denen eine Spannung von entweder 0V oder 5V angelegt oder ausgegeben werden kann.
    Des Weiteren besitzt es sechs analoge Pins, welche auch auf Spannungsschwankungen reagieren können.
    Weitere wichtige Pins sind In-Pins, welche eine Spannung von 5V liefern können und Out-Pins, welche eine Verbindung zur Erdung herstellen.

    \section{Begriffe}
    \label{sec:begriffe}

    \subsection{Pulsewellenmodulation (PWM)}
    \label{subsec:pulsewellenmodulation-(pwm)}


    \section{Aufgabe 1 - Ampelsteuerung}
    \label{sec:aufgabe-1}

    Es soll eine Ampelsteuerung implementiert und getestet werden.
    Die Ampel wird mithilfe von drei LEDs, in den Farben Rot, Gelb und Grün, aufgebaut.
    Weiters soll die Steuerung folgende Funktionsweise implementieren:

    \textbf{Phase 1} soll die Ampel auf Rot setzen.
    D.h., die rote LED wird eingeschaltet.
    Dieser Zustand soll vier Sekunden lang gehalten werden.

    \textbf{Phase 2} soll zusätzlich zur roten LED die gelbe einschalten.
    Dieser Zustand soll eine Sekunde lang gehalten werden.

    \textbf{Phase 3} soll die rote sowie die gelbe LED ausschalten, während die Grüne eingeschaltet wird.
    Dieser Zustand soll vier Selunden lang gehalten werden.

    \textbf{Phase 4} soll die grüne LED ausschalten während die Gelbe eingeschaltet wird.
    Dieser Zustand soll eine Sekunde lang gehalten werden.
    Nach Ablauf der vier Sekunden soll die grüne LED erlöschen und der Ablauf bei Phase 1 neu gestartet werden.

    \subsection{Materialien}
    \label{subsec:A1-materialien}

    \begin{table}[h]
        \centering
        \caption{Aufgabe 1 - Verwendete Materialien}
        \label{tab:a1-materialien}
        \begin{tabular}{| l | l | l |}
            \hline
            Bezeichnung & Eigenschaften & Menge \\
            \hline
            Widerstand  & $150\Omega$   & 3     \\
                        & Braun - Grün - Braun - Gold & \\
            LED & Rot & 1 \\
            LED & Gelb & 1 \\
            LED & Grün & 1 \\
            Mikrocontroller & Arduino Uno R3 & 1 \\
            \hline
        \end{tabular}
    \end{table}

    \subsection{Vorbereitung}
    \label{subsec:A1-vorbereitung}

    Für den Schaltkreis müssen die Vorwiederstände für die LEDs berechnet werden.
    Folgende Angaben sind bekannt.

    \begin{itemize}
        \item Ausgangsspannung der Pins des Mikrocontrollers: $V_{out} = 5V$
        \item Diodenspannung der LEDs: $U_D = 2V$
        \item Diodenstrom der LEDs: $I_D = 15mA$
    \end{itemize}

    Zur Berechnung wird folgender Stromkreis angenommen.

    \begin{figure}[h]
        \centering
        \includegraphics[height=0.4\textheight]{pictures/A1.png}
        \caption{Stromkreis A1}
        \label{fig:stromkreis-a1}
    \end{figure}

    Die Formel zur Berechnung des Stroms durch eine Diode ist bekannt.

    \begin{equation}
        I_D =  \frac{1}{R_d} * (U - U_D) \label{eq:diodenstrom}
    \end{equation}

    Damit kann der benötigte Vorwiderstand berechnet werden.

    \begin{equation}
        \begin{align}
            I_D =  \frac{1}{R_D} * (U - U_D) \Rightarrow \\
            R_D = \frac{1}{I_D} * (U - U_D) \\
            = \frac{1}{15mA} * (5V - 2V) \\
            = 200\Omega
        \end{align}
        \label{eq:equation-a1}
    \end{equation}

    Nachdem in der E6 Reihe keine $200\Omega$ Widerstände vorhanden sind, wurden $150\Omega$ gewählt.
    Diese Wahl erfolgt aus folgenden Überlegungen.

    \begin{enumerate}
        \item Laut angabe benötigt die LED $2V$ um zu schalten.
        \item Es existieren die Widerstände $150\Omega$ und $220\Omega$ in der E6-Reihe
        \item Bei einem Widerstand von $220\Omega$ würden, laut Gleichung \ref{eq:diodenstrom}, $I_D = \frac{3V}{220\Omega} = 13,636mA$ durch die LED fließen.
        \item Bei einem Widerstand von $150\Omega$ würden, laut Gleichung \ref{eq:diodenstrom}, $I_D = \frac{3V}{150\Omega} = 20mA$ durch die LED fließen.
    \end{enumerate}
    Da die Leuchtkraft einer LED von der Stromstärke abhängt und der Maximalstrom bei $20mA$ liegt, wurden $150\Omega$ gewählt.
    Bei dieser Größe wird die LED bei theoretisch voller Leuchtkraft betrieben ohne die Lebensdauer markant zu verkürzen.

    \subsection{Praktikumsaufgabe}
    \label{subsec:praktikumsaufgabe}

    Der Stromkreis wurde laut Abbildung \ref{fig:stromkreis-a1} implementiert.
    Die Implementierung wird in Abbildung \ref{fig:implementierung-a1} gezeigt.

    \begin{figure}[h]
        \centering
        \includegraphics[width=\textwidth]{pictures/a1-praktik.png}
        \caption{Implementiert Stromkreis von Aufgabe 1}
        \label{fig:implementierung-a1}
    \end{figure}

    Wie in Abbildung \ref{fig:implementierung-a1} gezeigt, wurden die Pins 13, 12 und 8 gewählt.
    Für die Wahl wurden die digitalen Pins herangezogen, da eine analoge Ausgabe nicht erforderlich war.
    Die Pins die mit einer Tilde ($\sim$) markiert sind, sind in der Lage ein PWM-Signal zu liefern.
    Da ein solches Signal nicht benötigt wird, wurden auch diese ausgeschlossen.
    Von oben nach unten betrachte, wurden nun drei Pins ausgewählt, diese sind 13, 12, und 8.
    Die genannten Pins werden, im Code, als digitale Output-Pins konfiguriert.

    Im nachfolgendem Text wird der Programmcode der Aufgabe erläutert.
    Einzelne Teile des Codes werden ausgewählt und beschrieben.
    Am Ende dieser Sektion befindet sich der vollständige Programmcode.

\begin{lstlisting}[language=C,label={lst:a1-konstanten}, caption={Konstanten für Aufgabe 1}]
const int PIN_RED = 13;
const int PIN_YELLOW = 12;
const int PIN_GREEN = 8;
\end{lstlisting}

    In Listing \ref{lst:a1-konstanten} werden die Pins definiert.
    Durch das Anlegen von Konstanten für die Pins, kann der Code aussagekräftiger gestaltet werden.
    Des weiteren entsteht dadurch die Möglichkeit, andere Pins zu verwenden, ohne den gesamtem Code durchgehen zu müssen.
    Es reicht die Nummer an einer Stelle zu ändern.

\begin{lstlisting}[language=C,label={lst:a1-setup}, caption={Setup für Aufgabe 1}]
void setup()
{
  pinMode(PIN_RED, OUTPUT);
  pinMode(PIN_YELLOW, OUTPUT);
  pinMode(PIN_GREEN, OUTPUT);
}
\end{lstlisting}

    In Listing \ref{lst:a1-setup} werden die Pins angelegt und konfiguriert.
    Wie oben beschrieben werden die Pins als Output-Pins, d.h., als Spannungsquelle, angelegt.

\begin{lstlisting}[language=C,label={lst:a1-onoff}, caption={On- und Off Methoden für Aufgabe 1}]
void off(int pin) {
  digitalWrite(pin, LOW);
}

void on(int pin) {
  digitalWrite(pin, HIGH);
}
\end{lstlisting}

    In Listing \ref{lst:a1-onoff} werden die Pins entweder ausgeschalten (LOW) oder eingeschalten (HIGH).
    Durch das Verwenden dieser Methoden ist es einfacher, die Phasen zu definieren bzw., im Code zu erkennen.

\begin{lstlisting}[language=C,label={lst:a1-loop}, caption={Programmschleife für Aufgabe 1}]
void loop()
{
  off(PIN_YELLOW);
  on(PIN_RED);
  delay(4 * 1000);

  on(PIN_YELLOW);
  delay(1 * 1000);

  off(PIN_RED);
  off(PIN_YELLOW);
  on(PIN_GREEN);
  delay(4 * 1000);

  off(PIN_GREEN);
  on(PIN_YELLOW);
  delay(1 * 1000);
}
\end{lstlisting}

    In Listing \ref{lst:a1-loop} werden die einzelnen Phasen implementiert.
    Nach jeder Leerzeile, d.h., nach Zeile 6, 9, und 14 beginnt jeweils eine neue Phase.

    Zu Beginn wird Phase 1 implementiert.
    Diese Schaltet die gelbe LED, Pin 12, aus, falls vorher Phase 4 aktiv war und schaltet die rote LED, Pin 13, an.
    Danach wird die $delay(x)$ Funktion aufgerufen, welche die Programmausführung für $x$ Millisekunden unterbricht.
    Die Angabe von $x$ als Berechnung aus $Sekunde * 1000$ wurde gewählt, um im Code besser erkenntlich zu machen, dass es sich um Millisekunden handelt.
    Der Programmfluss wird daher für vier Sekunden unterbrochen.

    Dann folgt Phase 2, welche zusätzlich zur Roten auch die gelbe LED einschaltet.
    Der Programmfluss wird für eine Sekunde unterbrochen.

    Es folgt Phase 3.
    Die Rote und die gelbe LED werden ausgeschaltet.
    Die Grüne, Pin 8, wird eingeschaltet.
    Der Programmfluss wird für weitere vier Sekunden unterbrochen.

    Schlussendlich folgt Phase 4.
    Es wird die grüne LED wieder ausgeschaltet, während die Gelbe aktiv wird.
    Es folgt wieder eine Unterbrechung des Programms für eine Sekunde.
    Am Ende der $loop()$ Funktion wird sie wieder von Beginn an, d.h., von Phase 1 aus, ausgeführt.


    \begin{lstlisting}[language=C,label={lst:a1-programmcode}, caption={Vollständiger Programmcode für Aufgabe 1}]
const int PIN_RED = 13;
const int PIN_YELLOW = 12;
const int PIN_GREEN = 8;

void setup()
{
  pinMode(PIN_RED, OUTPUT);
  pinMode(PIN_YELLOW, OUTPUT);
  pinMode(PIN_GREEN, OUTPUT);
}

void loop()
{
  off(PIN_YELLOW);
  on(PIN_RED);
  delay(4 * 1000);

  on(PIN_YELLOW);
  delay(1 * 1000);

  off(PIN_RED);
  off(PIN_YELLOW);
  on(PIN_GREEN);
  delay(4 * 1000);

  off(PIN_GREEN);
  on(PIN_YELLOW);
  delay(1 * 1000);
}

void off(int pin) {
  digitalWrite(pin, LOW);
}

void on(int pin) {
  digitalWrite(pin, HIGH);
}
    \end{lstlisting}

    \section{Aufgabe 2 - Reaktionsspiel}
    \label{sec:aufgabe-2---reaktionsspiel}

    Dummy Text

    \addcontentsline{toc}{section}{References}
    \printbibliography
\end{document}
