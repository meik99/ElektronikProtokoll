\section{Aufgabe 4 - Wertbestimmung von Widerstand und Kondensator}
\label{sec:aufgabe-4---wertbestimmung-von-widerstand-und-kondensator}

In dieser Aufgabe geht es um Widerstände und Kondensatoren.
Diese sind Grundbausteine der Elektrotechnik.
Neben diversen Berechnungen sind folgende Fragen zu beantworten.

\begin{itemize}
    \item Können Unterschiede zwischen Messungen am Mikrocontroller und manuellen Methoden gefunden werden?
    \item Wenn Ja, welche und warum?
    \item Mit welchen Prinzipien und Überlegungen wurden die Widerstände gemessen?
    \item Welche Abweichungen ergeben sich beim Messen bekannter Kondensatoren?
\end{itemize}

\subsection{Materialien}
\label{subsec:a4-materialien}

\begin{table}[h]
    \centering
    \caption{Aufgabe 4 - Verwendete Materialien}
    \label{tab:a4-materialien}
    \begin{tabular}{| l | l | l |}
        \hline
        Bezeichnung & Eigenschaften & Menge \\
        \hline
        Widerstand  & $10k\Omega$   & 2     \\
        & Braun - Schwarz - Orange - Gold & \\
        Widerstand & unbekannt & k.A. \\
        Mikrocontroller & Arduino Uno R3 & 1 \\
        \hline
    \end{tabular}
\end{table}

In der Tabelle \ref{tab:a4-materialien} sind unbekannte Widerstände in nicht angegebener Menge gelistet.
Damit sind Testwiderstände gemeint, welche durch das Testen an der Schaltung gemessen werden können.
Beim Versuch an dieser Schaltung sind die Widerstandswerte normalerweise bekannt, um die Richtigkeit der Schaltung zu bestätigen.
Die Werte und Anzahl der verwendeten Widerstände sind jedoch nicht relevant.

\subsection{Vorbereitung}
\label{subsec:a4-vorbereitung}

\subsubsection{Aufgabe 1}

In dieser Aufgabe sind Widerstände an Hand ihrer Farbcodes zu bestimmen.
Drei Widerstände sind gegeben.
Anhand der Tabelle in Sektion \ref{tab:farbcodierung-von-widerständen} kann der Wert bestimmt werden.

\begin{table}[ht]
    \centering
    \caption{Widerstandswerte}
    \label{tab:a4-widerstandswerte}
    \begin{tabular}{| l | l | l | l | l | l |}
        \hline
        Ring 1 & Ring 2 & Ring 3 & Ring 4 & Widerstandswert & Toleranz \\
        \hline
        Gelb & Violett & Rot & Gold & $4,7k\Omega$ & $\pm5\%$ \\
        Rot & Weiß & Grün & Gold & $3,9M\Omega$ & $\pm5\%$ \\
        Blau & Grau & Rot & Silber & $6,8k\Omega$ & $\pm10\%$ \\
        \hline
    \end{tabular}
\end{table}

\subsubsection{Aufgabe 2}


\subsection{Praktikumsaufgabe}
\label{subsec:a4-praktikumsaufgabe2}

\subsection{Fehlerdiskussion}
\label{subsec:a4-fehlerdiskussion}

\subsection{Zusammenfassung}
\label{subsec:a4-zusammenfassung2}
