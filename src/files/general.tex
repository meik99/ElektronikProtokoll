\section{Einleitung}
\label{sec:einleitung}

Das Wissen um die Funktionsweise eines Mikrocontrollers ist wesentlicher Bestandteil eines Informatik-Studiums.
Im Zuge Dieses werden in diesem Protokoll verschiedene Experimente beschrieben und evaluiert.
Insgesamt werden sieben Experimente begutachtet und diskutiert, welche im Zeitraum vom 28.September 2020 und 2.Oktober 2020 durchgeführt wurden.


\section{Verwendete Materialien}
\label{sec:verwendete-materialien}

Jede Sektion gibt die in ihr verwendeten Materialien mitsamt ihrer Stückzahl an.
In dieser Sektion werden alle Materialien im Allgemeinen gelistet und beschrieben.

\subsection{LED}
\label{subsec:led}
Eine Leuchtdiode (LED) ist ein Bauelement der Elektronik.
Wird sie von elektrischen Strom durchflossen beginnt sie Licht auszustrahlen.
Die Wellenlänge, d.h., die Farbe des Lichts sowie ob es für das menschliche Auge sichtbar ist oder nicht, hängt von den benutzten Materialien im inneren der LED ab.
Die in der LED verwendeten Materialien sind für dieses Protokoll nicht weiter von Bedeutung.
Für die beschriebenen Aufgaben wurden die folgenden LEDs verwendet.

\begin{table}[h]
    \centering
    \caption{LEDs - Farben, Flussspannung, Maximalstrom \cite{led-elektrische-eigenschaften}}
    \label{tab:leds-farben-und-elemente}
    \begin{tabular}{| l | l | l |}
        \hline
        Farbe & Durchflussspannung & Maximalstrom \\
        \hline
        Rot   & 1,6V - 2,2V        & 20mA         \\
        Gelb  & 1,9V - 2,5V        & 20mA         \\
        Grün  & 1,9V - 2,5V        & 20mA         \\
        \hline
    \end{tabular}
\end{table}

\subsection{Widerstände}
\label{subsec:widerstände}

Widerstände werden verwendet, um einen Spannungsabfall in einem Stromkreis zu verursachen.
Damit kann die Stromstärke in einem Stromkreis begrenzt beziehungsweise verringert werden.
Oft wird bei der Verwendung von LEDs ein sogenannter Vorwiederstand verwendet, um die Stromstärke soweit abzusenken, dass die LED nicht beschädigt wird.

In den Aufgaben wurden Widerstände der sogenannten E6-Reihe verwendet.
E-Reihen sind normierte Widerstandsgrößen, wobei die Zahl die Stufen zwischen den Potenzen angibt.
D.h., bei der E6-Reihe sind sechs verschiedene Widerstandsgrößen zwischen $10\Omega$ und $100\Omega$, zwischen $100\Omega$ und $1000\Omega$, u.s.w. bis zum oberen Limit von $10M\Omega$.

Die verwendeten Widerstände sind Farbcodiert, um sie voneinander unterscheiden zu können.
Die Farbkodierung von Widerständen wird in den Aufgaben, in denen sie verwendet werden, angegeben.
Die Bedeutung der Codierung der verwendeten Widerstände kann in Tab. \ref{tab:farbcodierung-von-widerständen} abgelesen werden.

\begin{table}
    \begin{adjustwidth}{-2cm}{}
        \caption{Farbcodierung von Widerständen}
        \label{tab:farbcodierung-von-widerständen}
        \begin{tabular}{| l | l | l | l | l |}
            \hline
            Farbe   & 1.Ring (10er Stelle) & 2.Ring (1er Stelle) & 3.Ring (Multiplikator) & 4.Ring (Toleranz) \\
            \hline
            Silber  & -                    & -                   & -                      & $\pm 10\%$        \\
            Gold    & -                    & -                   & $0.1$                  & $\pm 5\%$         \\
            Braun   & 1                    & 0                   & $10$                   & $\pm 1\%$         \\
            Rot     & 2                    & 2                   & $100$                  & $\pm 2\%$         \\
            Gelb    & 4                    & 4                   & $10\ 000$              & -                 \\
            Grün    & 5                    & 5                   & $100\ 000$             & $\pm 0.5\%$       \\
            Blau    & 6                    & 6                   & $1\ 000\ 000$          & $\pm 0.25\%$      \\
            Violett & 7                    & 7                   & $10\ 000\ 000$         & $\pm 0.1\%$       \\
            Grau    & 8                    & 8                   & $100\ 000\ 000$        & $\pm 0.05\%$      \\
            Weiß    & 9                    & 9                   & $1\ 000\ 000\ 000$     & -                 \\
            \hline
        \end{tabular}
    \end{adjustwidth}
\end{table}

\subsection{Taster}
\label{subsec:taster}

\subsection{Mikrocontroller \cite{arduino-r3}}
\label{subsec:mikrocontroller}

Mikrocontroller werden verwendet, um komplexere Logik in eine elektronische Schaltung zu integrieren.
Die in diesem Protokoll verwendete Mikrocontroller sind Mikrocontroller vom Typ Arduino Uno Rev 3.
Diese Mikrocontroller besitzen 14 digitale Pins, an denen eine Spannung von entweder 0V oder 5V angelegt oder ausgegeben werden kann.
Des Weiteren besitzt es sechs analoge Pins, welche auch auf Spannungsschwankungen reagieren können.
Weitere wichtige Pins sind In-Pins, welche eine Spannung von 5V liefern können und Out-Pins, welche eine Verbindung zur Erdung herstellen.

\subsection{Logikbausteine}
\label{subsec:logikbausteine}

Um verschiedene Logikgatter in Schaltungen verwenden zu können, ohne komplizierte Schaltkreise bauen zu müssen, werden Logikbausteine verwenden.
Diese Bausteine implementieren verschiedene Gatter.
D.h., bei zwei Eingängen $a$ und $b$, wird durch das Anwenden der Funktion, die der Baustein implementiert, der Ausgang $y$ erzeugt.
Diese Funktionen bilden logische Operationen ab.
Die Erklärung von logischen Funktionen würden über den Umfang dieses Protokolls hinausgehen und werden daher nicht weiter erklärt.
In Tabelle \ref{tab:bausteine-und-deren-logische-funktionen} sind die verwendeten Bausteine und die Funktionen die sie implementieren gelistet.

\begin{table}[ht]
    \centering
    \caption{Bausteine und deren logische Funktionen}
    \label{tab:bausteine-und-deren-logische-funktionen}
    \begin{tabular}{| l | l |}
        \hline
        Bausteinbezeichnung & Funktion \\
        \hline
        74HC00              & NAND     \\
        74HC02              & NOR      \\
        74HC08              & AND      \\
        74HC32              & OR       \\
        74HC86              & XOR      \\
        \hline
    \end{tabular}
\end{table}

\subsection{Kondensatoren}
\label{subsec:kondensatoren}

\subsection{Operationsverstärker}
\label{subsec:operationsverstaerker}

\subsection{Motortreiber}
\label{subsec:motortreiber}

\subsection{Ultraschallsensor}
\label{subsec:ultraschallsensor}

\subsection{7-Segment Anzeige}
\label{subsec:7-segment-anzeige}

\section{Begriffe}
\label{sec:begriffe}

\subsection{Pulsewellenmodulation (PWM)}
\label{subsec:pulsewellenmodulation-(pwm)}

\subsection{PullUp}
\label{subsec:pullup}

\subsection{Baudrate}
\label{subsec:baudrate}

\subsection{Wahrheitstabelle}
\label{subsec:wahrheitstabelle}
