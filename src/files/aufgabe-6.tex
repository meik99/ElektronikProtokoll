\section{Aufgabe 6 - Motorsteuerung mit Nähreungssensor}
\label{sec:aufgabe-6---motorsteuerung-mit-naehreungssensor}

In dieser Aufgabe soll mithilfe ein Motor angesteuert werden.
Um die Geschwindigkeit des Motors zu steuern, wird eine Regulierung mittels PWM-Signal realisiert.
Der Motor soll weiters auf einen Ultraschallsensor reagieren, welche die Entfernung zu Objekten misst.
Nähert sich ein Objekt auf weniger als 30cm, so soll der Motor ausgeschaltet, bzw.\ seine Geschwindigkeit auf 0 reduziert, werden.
Eine 7-Segment Anzeige soll die derzeitige Geschwindigkeitsstufe des Motors anzeigen und mittels Schieberegister gesteuert werden.

Zwei Taster sollen einerseits die Laufrichtung des Motors und andererseits die Geschwindigkeit ändern.
Wird Taster A betätigt, soll sich die Geschwindigkeit, mit der sich der Motor gegen den Uhrzeigersinn dreht, um eins erhöht werden.
Dreht sich der Motor bei Betätigung des Tasters im Uhrzeigersinn, so wird die Laufrichtung und Geschwindikeit entsprechend geändert.
Wird Taster B gedrückt, soll das selbe Verhalten für die andere Laufrichtung implementiert werden.

Die Geschwindikeiten sollen hierbei zwischengespeichert und wiederhergestellt werden.
Es soll daher das folgende Beispiel möglich sein.

\begin{itemize}
    \item Motor dreht sich mit Geschwindigkeit 3 im Uhrzeigersinn
    \item Taster A wird betätigt
    \item Motor dreht sich mit Geschwindigkeit 1 gegen den Uhrzeigersinn
    \item Taster A wird betätigt
    \item Motor dreht sich mit Geschwindigkeit 2 gegen den Uhrzeigersinn
    \item Taster B wird betätigt
    \item Motor dreht sich mit Geschwindigkeit 4 im Uhrzeigersinn
    \item Taster B wird betätigt
    \item Motor dreht sich mit Geschwindigkeit 5 im Uhrzeigersinn
    \item Taster A wird betätigt
    \item Motor dreht sich mit Geschwindigkeit 3 gegen den Uhrzeigersinn
\end{itemize}

Würde die Geschwindigkeit mittels Tastendruck auf $> 9$ gebracht werden wird sie wieder auf 0 zurückgesetzt.

\subsection{Materialien}
\label{subsec:a6-materialien}

\begin{table}[h]
    \centering
    \caption{Aufgabe 5 - Verwendete Materialien}
    \label{tab:a5-materialien}
    \begin{tabular}{| l | l | l |}
        \hline
        Bezeichnung & Eigenschaften & Menge \\
        \hline
        7-Segement Anzeige & & 1 \\
        Ultraschallsensor & HC-SR04 & 1 \\
        Schieberegister & 74HC595 & 1 \\
        Taster & 4 Polig & 2\\
        Motortreiber & L293D & 1 \\
        LED & Rot & 1 \\
        LED & Grün & 1 \\
        Widerstand & $150\Omega$ & 10 \\
        & Braun - Grün - Braun - Gold & \\
        Widerstand & $10k\Omega$ & 2 \\
        & Braun - Schwarz - Orange - Gold & \\
        Diode & & 2 \\
        DC-Motor & & 1 \\
        Mikrocontroller & Arduino Uno R3 & 1 \\
        \hline
    \end{tabular}
\end{table}

\subsection{Vorbereitung}
\label{subsec:a6-vorbereitung}

\subsection{Praktikumsaufgabe}
\label{subsec:a6-praktikumsaufgabe}

\subsection{Fehlerdiskussion}
\label{subsec:a6-fehlerdiskussion}

\subsection{Zusammenfassung}
\label{subsec:a6-zusammenfassung}
